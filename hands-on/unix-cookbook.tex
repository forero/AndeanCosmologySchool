\documentclass[12pt]{article}
\usepackage{enumerate}
\usepackage[hmargin=2.0cm,vmargin=1cm]{geometry}
\usepackage[utf8]{inputenc}
\usepackage{graphicx}
\usepackage{float}
\usepackage{cite}
\usepackage{natbib}
\usepackage{color}
\usepackage{hyperref} 


\title{\begin{LARGE}{ \textsl{Andean Cosmology School \\
Unix basics cookbook}}
\end{LARGE}}

\author{Juan Nicol\'as Garavito Camargo}

\begin{document}
\date{June 2, 2015}

\maketitle

\section{The Terminal}

The terminal is the place in which you are going to 
interact with the computer. 

\subsection{Opening the terminal}

You can open the terminal by simply going 
to the ubuntu search button (usually located
at the top left), type terminal and click 
on the terminal icon. \\

A faster way in Ubuntu is typing \verb"Ctrl + Alt + t".

\subsection{Terminal commands}

\textbf{Directories:}

The first thing that you want to check 
is who is logged into the computer.\\

1. \verb+whoami+

{\color{blue} \verb+whoami+ print the username}\\



2. \verb+pwd+  

{\color{blue} \verb+pwd+ print the current/working directory}\\

3. \verb+ls+

{\color{blue} \verb+ls+ list the directory contents}\\

4. \verb+cd Documents+

{\color{blue} \verb+cd+ change the current directory and go inside the Documents
directory}\\

5. \verb+ cd ..+ 

{\color{blue} \verb+cd ..+ go up in the directory}\\

6. \verb+mkdir escoan+

{\color{blue}} \verb+mkdir+ makes a new directory call \verb+escoan+.
\verb+mkdir+\\

7. \verb+cd escoan+\\

8. \verb+mkdir hadsonRF+

{\color{blue} make a new wrongly named directory \verb+hadsonRF+ }\\

9. \verb+rmdir hadsonRF+

{\color{blue} remove the \verb+hadsonRF+ directory}\\

10. \verb+mkdir handson+\\

11. \verb+cd handson+\\

12. \verb+mkdir sesion1+\\

13. \verb+cd sesion1+\\

\textbf{Downloading files:}\\

Now download some data to play with.

14. {\scriptsize{\verb+wget https://raw.githubusercontent.com/forero/AndeanCosmologySchool/master/hands-on/halocatalogue.tar.gz+}} \\

\textbf{Files:}

15. \verb+mkdir data+\\

16. \verb+mv halocatalogue.tar.gz data/+ 

{\color{blue} move \verb+halocatalogue+ to the data folder}\\

17. \verb+cd data+ \\

18. \verb+gunzip halocatalogue.tar.gz+ 

{\color{blue} unzip the \verb+halocatalogue.tar.gz+ file }\\

19. \verb+tar -xvf halocatalogue.tar+ 

{\color{blue} expands the file \verb+halocatalogue.tar+}\\

20. \verb+cp Millenium13 ../+ 

{\color{blue} copy the Millenium13 file to /sesion1}\\

21. \verb+less Millenium13+

{\color{blue} \verb+less+ shows the content of Millenium13}\\

22. \verb+q+

{\color{blue} \verb+q+ exit from Millenium13 file}\\

23. \verb+cat Millenium13+

{\color{blue} Displays all the Millenium13 content to stop press \verb"Ctrl+c"}\\

 
24. \verb+head Millenium13+ 

{\color{blue} Shows the first 10 lines of Millenium13, the \verb+-l+ flag
allows to change the number of lines.}\\

25. \verb+tail Millenium13+

{\color{blue} Shows the last 10 lines of Millenium13.}\\

\textbf{Exploring files:}\\

26. \verb+wc Millenium13+

{\color{blue} counts the number of lines, words, and bytes in Millenium13}\\
 
27. \verb+grep haloId MIllenium13+

{\color{blue} find the haloId word in Millenium13 and print the line in which it appears.}\\

28. \verb+awk '{print $1}' Millenium13 > col1.txt+

{\color{blue} print the first column of the Millenium13 file and redirects the ouptut to \verb+col1.txt+}\\

29. \verb"awk '{print $18+$19}' Millenium13 > xy.txt" \\

30. \verb+awk '{if ($19>100) print $0}'+

{\color{blue} print all lines in which column 19 is greater than 100.}\\

31. \verb+sed 's/haloId/haloID/g' Millenium13 > newID+

{\color{blue} changes haloId for haloID and redirects the ouputu to newID}\\

\textbf{Useful commands}

\verb+history+

\verb"Ctrl + r"

\verb+man+

\verb+Tab+


\section*{Scripts and text editors}

To open an emacs file:\\

\texttt{emacs} \verb" halomc.sh &"  \\

The $\&$ would open an external window without freezing
the terminal.\\

\textbf{Exercise:}

Write a script that makes the folllowing:\\ 

1. Select dark matter halos with masses below
\verb+m_crit200 < 5+ from the \verb+halocatalog.txt+. Write the 
$id, x, y, z, m\_crit200, vMax$ columns for this halos in a 
separate file
called \verb+haloslowmass.txt+.\\

2. Repeat step 1 for \verb+m_crit200 > 10+, the new file should
be called \verb+haloshighmass.txt+.\\

3. Change the \verb+.txt+ files to \verb+.csv+ files.  

Cut text:\\

\texttt{Ctrl+w}\\

Copy text:\\

\texttt{M+w}

Paste text:\\

\texttt{Ctrl+y}\\

Save:\\

\texttt{Ctrl+x+s}\\

Exit Emacs:\\

\texttt{Ctrl+x+c}\\

\textsc{Emacs offitial web site:}\\
\verb"https://www.gnu.org/software/emacs/"\\

\textsc{Usefull manual and tutorials}\\
\verb"https://www.gnu.org/software/emacs/manual/html_node/emacs/index.html"\\
\verb"http://www.drpaulcarter.com/cs/emacs.php"\\


\subsection{Making a script executable}

\verb"chmod u+x script.sh"
{\color{blue} chmod changes the file mode of the .sh file}

\subsection{Running a script}
 \verb+./script.sh+
\end{document} 
