
\documentclass[11pt]{amsart}
\usepackage{geometry} % see geometry.pdf on how to lay out the page. There's lots.
\geometry{a4paper} % or letter or a5paper or ... etc
% \geometry{landscape} % rotated page geometry
\parindent 0pt
\setlength{\parskip}{0.8cm plus4mm minus3mm}
\def \d {{\rm d}}
\def \kms {{\rm km s$^{-1}$}}
\def \Rvir{$  R_{\rm vir}$  }
\def \rvir {$  R_{\rm vir}$  }
\def \kmsMpc {{\rm km s$^{-1}$ Mpc$^{-1}$}}
\newcommand{\hmsun}{{\,\rm h^{-1}M}_\odot}
\newcommand{\hmpc}{{\,\rm h^{-1}Mpc}}

\title{Project: Press - Schechter Mass Function}
\author{Yehuda Hoffman}
\date{June 10, 2015} % delete this line to display the current date

%%% BEGIN DOCUMENT
\begin{document}

\maketitle
%\tableofcontents
%
%\section{}
%\subsection{}


Dear Students,

Here are a few comments, instructions and some hints on calculating the Press - Schechter (PS) mass function:

\paragraph{\bf PS mass function } \hspace{10pt} \\
Here is a reminder of the PS mass function that we have derived at class:
\begin{equation}
\label{eq:eta}
\eta(M,a)=\sqrt{2\over\pi}\  {\bar{\rho}\over M^2} \ {\delta{^L_c}(a)\over\sigma_M}\  \exp\big[- {(\delta{^L_c}(a))^2\over2\sigma{^2_M}}\big] 
\  \Big\vert {\d \ln \sigma_M \over\d \ln M } \Big\vert
\end{equation}

Calculate here the mass function for the present epoch ($a=1.0$) and for the WMAP set of cosmological parameters. The relevant parameters are:\begin{itemize}
  \item $\sigma_8= \sigma(R=8\hmpc)= 0.817$
  \item $\Omega_m = 0.279$
  \item $\Omega_\Lambda = 1 - \Omega_m$
\end{itemize}

See the note on the $h^{-1}$ scaling below.

\paragraph{\bf Analytical fit to the power spectrum } \hspace{10pt} \\
For the sake of the present calculation we shall use here an analytical fit the power spectrum. The fit is due to Max Tegmark (reference is missing):
\begin{equation}
\label{eq:Pk}
P(k) = A {k^n \over \Big\{ 1 + \Big[a k / \Gamma+ \big(b k / \Gamma \big)^{3/2} +  \big(c k / \Gamma \big)^2  \Big]^\nu\Big\}^{1 / \nu}}
\end{equation}
Where:
\begin{itemize}
  \item $n=1$
  \item $\Gamma = 0.21$
  \item $a = 6.4 \hmpc$
  \item $b = 3.0 \hmpc$
  \item $a = 1.7 \hmpc$
  \item $\nu = 1.13$
\end{itemize}

\paragraph{\bf Use of the $h$ scaling } \hspace{10pt} \\
Distances in cosmology depend on the value of $H_0$ (Hubble's constant) - recalling that in an unperturbed Friedmann universe we have $v=H_0\times {\rm distance}$, where the velocity ($v$) is the observable. It is customary  to parametrize the value of $H_0$ by the following expression: $H_0 = 100\  h\ {\rm km/s/Mpc }$. It follows that one can express masses and distances as if the value of $H_0$ is $100\   {\rm km/s/Mpc }$ and correct for the actual value by using units of $\hmsun$ (for   masses) and 
$\hmpc$ (for distances). In your calculations use the $h^{-1}$ scaling.

\paragraph{\bf Top-Hat Model: $\delta{^L_c}$ (for $a=1.0$) } \hspace{10pt}\\
You are supposed to calculate the value of  $\delta{^L_c}$ from the top hat model, assuming the standard $\Lambda$CDM model whose WMAP parameters are given above. To   enable you to work in parallel on the two assignments the following  is provided here:   $\delta{^L_c}=1.67484$ 

\paragraph{\bf Presentation of $\eta(M)$}  \hspace{10pt}\\
Recall that we have derived the mass function by the following expression: 
$$\eta(M)={\bar{\rho}\over M} {\partial F(>M)\over\partial M}$$
If we multiply the two sides of the equation by $M$ we get:
$$M \eta(M)={\bar{\rho}\over M}  {\partial F(>M)\over\partial \ln M}$$
Namely, it is the mass function defined per $\ln M$.
It is customary to present the mass function by $M \eta(M)$. Please do the same, in particular when you plot it.

\bigskip
If you have any problems with the calculations of the two projects, top hat model and the mass function, please do not hesitate to contact me by mail (hoffman@huji.ac.il). You can do it also after I'll be leaving the school.

Enjoy it,

Yehuda


\end{document}